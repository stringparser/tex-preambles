%%%%%%%%%%%%%%%%%%%%%%%%%%%%%%%%%%%%%%%%%%%%%%%%%%%%%%%%%%%%%%%%%%%%%%%%%%%%%%%%%%%%%%%%%%%%%%%%%%%%
%% CUSTOM COMMANDS
%%%%%%%%%%%%%%%%%%%%%%%%%%%%%%%%%%%%%%%%%%%%%%%%%%%%%%%%%%%%%%%%%%%%%%%%%%%%%%%%%%%%%%%%%%%%%%%%%%%%
%% To define a comand one writes
%%
%% \newcommand{<name>}[<# of arguments>]{ what makes the command on the arguments }
%%
%% Example 1: write with boldface and italic
%% \newcommand{\bfita}[#1]{ \textbf{ \textit{#1} } }
%% Usage: \bfita{this will be a text on italic and boldface}
%%
%% Example 2: write partial derivatives fast (used in the transcript of the lecture notes) 
%% \newcommand{\pder}[2]{ \frac{ \partial #1 }{ \partial #2 } }
%% Usage: \pder{f(x)}{x} = \frac{\partial f(x)}{\partial x}
%% 
%%%%%%%%%%%%%%%%%%%%%%%%%%%%%%%%%%%%%%%%%%%%%%%%%%%%%%%%%%%%%%%%%%%%%%%%%%%%%%%%%%%%%%%%%%%%%%%%%%%%

%%% Formating

% Traza
\newcommand{\Tr}{\text{Tr}}

% Codigo
\newcommand{\code}[1]{ \begin{center} \begin{boxedminipage}{0.8\textwidth} \begin{verbatim} #1
\end{verbatim} \end{boxedminipage} \end{center} }

\newcommand{\cbox}[2]{ \begin{center} \begin{boxedminipage}{0.8\textwidth} \textbf{#1} #2
\end{boxedminipage} \end{center} }

% CHorizontal line
\newcommand{\hl}{ \noindent \small \hspace{-0.4cm}$||$||||||||||||||||||||||| \\
\vspace{-0.1cm}}

% Cpart
\newcommand{\cpart}[1]{
\newpage 
\begin{center}
\Huge \textbf{Part #1} \\ \vspace{-0.5cm} |||||||| \\
\end{center}
}

% Example
\newcommand{\ej}{$\triangleright$ \uwave{EJEMPLO}\\}

% Comments on the script 
\newcommand{\comment}[1]{ \textcolor{blue}{ \texttt{\small \sloppy  #1} } }

% CParagrah: \uwave{} draws a wave underlining the text of the brackets
\newcommand{\cparagraph}[1]{ {\large \uwave{\textbf{#1}}} \\}

% CExample
\newcommand{\example}[1]{  { \uwave{\textbf{ \large Example}}: #1} \\ }
\newcommand{\examples}[1]{  { \uwave{\textbf{ \large Examples}}: #1} \\ }

% Custom notes CNote 
\newcommand{\note}[1]{ $\xrightarrow{\text{ \normalsize \textbf{#1}}}$}

% Custom boxedminipage (To box theorems, definitions or lemmas fast and easy)
\newcommand{\minibox}[1]{
\begin{center} \begin{boxedminipage}{15cm} #1 \end{boxedminipage} \end{center}
}


%%%%  Image stuff

% A simple image with caption
\newcommand{\img}[3]{ 
\begin{figure}[h!]
\centering
\includegraphics[#1]{#2}
\caption{#3}
\end{figure}

}

% A simple image without caption
\newcommand{\imgnc}[2]{ 
\begin{figure}[h!]
\centering
\includegraphics[#1]{#2}
\end{figure}

}

% CFigure reference
\newcommand{\figref}[1]{Figura \ref{#1}}

% CMissing image
\newcommand{\missing}[1]{
\begin{figure*}[h!]
\centering
\includegraphics[scale=0.5]{img/blank.png}
\caption{#1}
\end{figure*}
}

% Two cententered numbered images
\newcommand{\twocnimg}[6]{
\begin{figure}[h!]
\centering
\subfigure[#3]{\includegraphics[#1]{#2}} \hspace{0.2cm}
\subfigure[#6]{\includegraphics[#4]{#5}}
\end{figure}
}

% Three cententered numbered images
\newcommand{\threecnimg}[9]{
\begin{figure}[h!]
\centering
\subfigure[#3]{\includegraphics[#1]{#2}} \hspace{0.2cm}
\subfigure[#6]{ \includegraphics[#4]{#5}}
\subfigure[#9]{ \includegraphics[#7]{#8}}
\end{figure}
}

%%%%%%%% CMath stuff
\newcommand{\cancelc}[2]{\textcolor{blue}{\cancelto{\textcolor{blue}{#2}}{\textcolor{red}{#1}}}}
\newcommand{\rg}{\text{rg}}
\newcommand{\eqmark}{(\text{\textasteriskcentered})}
\newcommand{\tq}{\text{ }}  % A text space (text quad)
\newcommand{\asenh}{\text{asenh}}
\newcommand{\acosh}{\text{acosh}}
\newcommand{\atanh}{\text{atanh}}
%% Derivatives
\newcommand{\spder}[2]{\partial_{#2} #1}               % Small partial derivative
\newcommand{\tder}[2]{\frac{d #1}{d #2}}                % total derivative
\newcommand{\Tder}[2]{\frac{D #1}{D #2}}                % Total derivative
\newcommand{\pder}[2]{\frac{\partial #1}{\partial #2}} % Partial derivative with
\newcommand{\pdder}[2]{\frac{\partial^2 #1}{\partial #2^2}} % Partial derivative
\newcommand{\pdqer}[3]{\frac{\partial^2 #1}{\partial #2 \partial #3}} % Partial derivative
%% Derivatives with parenthesis
\newcommand{\pderp}[2]{\left(\frac{\partial #1}{\partial #2}\right)} % Partial derivative
\newcommand{\pdderp}[2]{\left(\frac{\partial^2 #1}{\partial #2^2}\right)} % Partial derivative
\newcommand{\pdqerp}[3]{\left(\frac{\partial^2 #1}{\partial #2 \partial #3}\right)} % Partial derivative


\newcommand{\fto}[1]{i=1,\ldots, #1}
\newcommand{\Eqref}[1]{Eq.\eqref{#1}}                  % Formatted equation reference
\newcommand{\eqnote}[2]{\stackrel{ \stackrel{#2}{\substack{\makebox[0pt]{\rotatebox{-90}{$\to$}}} } }{#1} }
% Dirac notation
\newcommand{\bra}[1]{\langle #1|}
\newcommand{\ket}[1]{|#1\rangle}
\newcommand{\braket}[2]{\left\langle #1|#2 \right\rangle}
\newcommand{\mean}[1]{\left\langle #1 \right\rangle}

% Environments
% \newenvironment{name}[num]{before}{after}


%%%%%%%%%%%%%%%%%%%%%%%%%%%%%%%%%%%%%%%%%%%%%%%%%%%%%%%%%%%%%%%%%%%%%%%%%%%%%%%%%%%%%%%%%%%%%%%%%%%%
%% Tables

% Fast array

\newcommand{\farray}[3]{ 
\begin{table}[h!]
\centering
\begin{tabular}{#1} \hline
 #3
\end{tabular}
\caption{#2}
\end{table}
}


% Two Fast array with common caption
\newcommand{\twofarray}[6]{ 
\begin{table}[h!] \centering 
\begin{tabular}{ccc} 
\begin{tabular}{#1}  
#3 
\end{tabular}
& 
\hspace{0.1cm} 
&
\begin{tabular}{#4}
 #6  
\end{tabular} \\ & & \\
(a) & & (b) \\
\end{tabular}
\caption{(a) #2 (b) #5}
\end{table}
}

% Two Fast ARRAY With Common Caption and Label

\newcommand{\twofarrayl}[8]{
\begin{table}[h!] 
\centering

\begin{tabular}{ccccc} 
(a)  & 
\begin{math}
\begin{array}{#1}  
\hline #4 \\ \hline
\end{array}
\label{#3}
\end{math} 
  & \hspace{0.1cm} & (b) &
\begin{math}
\begin{array}{#5} \hline
 #8 \\ \hline
\end{array}
\label{#7}
\end{math} \\
\end{tabular}
\caption{(a) #2. (b) #6.}
\end{table}
}

% Two Fast array with no common caption
\newcommand{\twofarrayncc}[6]{
\begin{table}[h!] 
\centering
\begin{tabular}{ccccc} 
#2  & 
\begin{math}
\begin{array}{#1}  
\hline #3 \\ \hline
\end{array}
\end{math} & \hspace{0.1cm} & 
#5 &
\begin{math}
\begin{array}{#4} \hline
 #6 \\ \hline
\end{array}
\end{math} \\
\end{tabular}
\end{table}
}

% Two Fast ARRAY with No Common Caption and Label

\newcommand{\twofarraynccl}[8]{
\begin{table}[h!] 
\centering

\begin{tabular}{ccccc} 
(a)  & 
\begin{math}
\begin{array}{#1}  
\hline #4 \\ \hline
\end{array}
\label{#3}
\end{math} 
  & \hspace{0.1cm} & (b) &
\begin{math}
\begin{array}{#5} \hline
 #8 \\ \hline
\end{array}
\label{#7}
\end{math} \\
\end{tabular}
\caption{(a) #2. (b) #6.}
\end{table}
}

%% ENVIRONMENTS

\newenvironment{king}
{\rule{1ex}{1ex}\hspace{\stretch{1}}}
{\hspace{\stretch{1}}\rule{1ex}{1ex}}

\newenvironment{freemath}{\begin{center} \begin{math}}{\end{math} \end{center}}
