%\documentclass[9pt,blue,compress]{beamer}
\mode<presentation>
%% --- The Standard Packages ---
% \usepackage{beamerthemesplit}
\usepackage{graphicx}
\usepackage{latexsym}
\usepackage{makeidx}

%% FUENTES
\usepackage[utf8]{inputenc} % Encoding LINUX
%\usepackage[spanish]{babel} % Tildes y la ñ española
%\usepackage{textcomp}
\sloppy % Mejor división de palabras

\usepackage{color}
% \color{``texto o formula''} -> Cambia el color a partir de este punto
% \texcolor{``color''}{texto o fórmula}
% \pagecolor{``black``} -> Cambia el color de la página a partir de este momento
% \definecolor{rosa}{rgb}{1,0.5,0.5} -> Definir un color personalizado en rgb

%% FUENTES MATEMÁTICAS
\usepackage{amsfonts}
\usepackage{amsmath}
\usepackage{amssymb}
\usepackage{amsthm}
\usepackage{dsfont}
 
%\usepackage{mathpazo}
%% --- Tablas largas ---
% Con este paquete se pueden poner tablas que sean muy grandes, utilizando /begin{longtable} en
% lugar de /begin{table} ---
\usepackage{longtable}
%% --- Landscape ---
% \usepackage{lscape}

%% --- Gricos ---
% Se ponen los paquetes necesarios para introducir imágenes.
\usepackage{psboxit}
\usepackage{subfigure}
% \usepackage{caption}
\usepackage{ulem}
\usepackage{cancel}
% --- Unidades ---
%\usepackage[squaren,Gray]{SIunits}


%%%%%%%%%%%%%%%%%%%%%%%% Cabeceras bonitas. 

\usepackage{calc}

% IZQUIERDA
%------------------------
% /fancyhead[]{} 
%	-L-> Izquierda
%	-R-> Derecha
%	-C-> Centrado
%	-O-> En un número de página impar, va acompañado de una de las tres posiciones anteriores
%(OL,OR,OC)
%	-E-> En un número de página par, va acompañado de una de las tres posiciones anteriores
%(EL,ER,EC)
%\fancyhead[L]{Escuela de Verano}

% CENTRO
%\fancyhead[R]{Javier Carrillo Milla}
% DERECHA
%\fancyhead[R,L]{\thepage}%Se puede poner un encabezado igual tanto a derecha como a izquierda
%Linie oben



%%%%%%%%%%%%%%%%%%%%%%%%%%%%%%%%%%%%%%%%%%%%%%%%%%%%%%% Pi� de p�gina
%\fancyfoot[L]{}
%\fancyfoot[LO]{\today}
 %AQUI SE PONE POR DEFECTO LA PAGINA, EN EL PIE DE
%PAGINAA LA DERECHA

%\renewcommand{\footrulewidth}{0.5pt}
\renewcommand{\thefootnote}{[\arabic{footnote}]} 

\usepackage{paralist}

%%%%%%%%%%%%%%%%%%%%%%%%%%%%%%%%%%%%%%%%%%%%%%%%%%%%%%% Bibliograf�a
% \usepackage[square,authoryear,sort&compress]{natbib}

%% Custom bibliography style defined using the Makebst utility
%% that comes with the natbib package.

% \bibliographystyle{bib/bibleo}

%%%%%%%%%%%%%%%%%%%%%%%%%%%%%%%%%%%%%%%%%%%%%%%%%%%%%%% Comandos personalizados
% Estos paquetes se usan para poner nombre a diversos objetos.
%\addto\captionsspanish{%
%  \def\prefacename{Prefacio}%
%  \def\refname{Referencias}%
%  \def\abstractname{Abstract}
%  \def\bibname{Bibliograf\'{\i}a}%
%   \def\chaptername{Part}%
%  \def\appendixname{Ap\'endice}%
%  \def\listfigurename{\'Indice de figuras}%
  %\def\listtablename{\'Indice de cuadros}%
%  \def\listtablename{Table index}%
%  \def\indexname{\'Indice alfab\'etico}%
%  \def\figurename{Figura}%
  %\def\tablename{Cuadro}%
%  \def\tablename{Table}%
%  \def\partname{Parte}%
%  \def\enclname{Adjunto}%
%  \def\ccname{Copia a}%
%  \def\headtoname{A}%
%  \def\pagename{P\'agina}%
%  \def\seename{v\'ease}%}
%  \def\alsoname{v\'ease tambi\'en}%
%  \def\proofname{Demostraci\'on}%
%  \def\glossaryname{Glosario}
%}

% Boxedminipage
 \usepackage{boxedminipage}
 \setlength{\fboxrule}{1pt}
 \setlength{\fboxsep}{3pt}

%\usetheme[hideothersubsections,width=1.7cm]{PaloAlto}
\useinnertheme{circles}
\setbeamertemplate{blocks}[rounded][shadow=true]
%\logo{\includegraphics[scale=0.3]{img/logo_lmu.pdf}}