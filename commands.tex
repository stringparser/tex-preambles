%%%%%%%%%%%%%%%%%%%%%%%%%%%%%%%%%%%%% COMANDOS PROPIOS

% TIPOGRAFÍAS

\newcommand{\tbbu}[1]{\noindent $\bullet$ \textbf{#1}:} % BOLDFACE & UNDERLINED
\newcommand{\te}[1]{\emph{#1}} % EMPHASIZE
\newcommand{\tu}[1]{\underline{#1}} % UNDERLINE
\newcommand{\tb}[1]{\textbf{#1}} % BOLDFACE
\newcommand{\hlinebreak}{\noindent |||||||||||||||||||||||||||||||||||||||||||||||||||| \\ }
\newcommand{\boxtext}[1]{
\hlinebreak
\noindent #1 \\
\hlinebreak
} 
\newcommand{\boxtextit}[1]{
\hlinebreak
\noindent \textit{#1} \\
\hlinebreak
} 
\newcommand{\tit}[1]{\textit{#1}} % ITALIC

% Boxed Title
\newcommand{\boxedtitle}{
\begin{boxedminipage}{17.1cm}
\begin{boxedminipage}{16.75cm} 
\vspace{6cm}
\maketitle  
\vspace{6cm}
\end{boxedminipage}
\end{boxedminipage}
}

% Centered Boxed Minipage, custom width

\newcommand{\cbox}[1]{
\begin{center}
\begin{boxedminipage}{15cm} #1 \end{boxedminipage}
\end{center}
}

% Teoremas, definiciones y tal

\newcommand{\definicion}[1]{
\cbox{ \textbf{Definición}: #1 }
}

\newcommand{\teorema}[1]{
\cbox{ \textbf{Teorema}: #1 }
}

\newcommand{\proposicion}[1]{
\cbox{ \textbf{Proposicion}: #1 }
}

\newcommand{\Lema}[1]{
\cbox{ \textbf{Lema}: #1 }
}

% PARA EL MATH MODE

% Tipografías

\newcommand{\cali}[1]{\mathcal{#1}}

\newcommand{\rblr}[1]{\left( #1 \right)}    % Round brackets, open brackets, or parentheses:  ( )
\newcommand{\cblr}[1]{\left\{ #1 \right\}}  % Curly brackets or braces:  { }
\newcommand{\sblr}[1]{\left[ #1 \right]}    % Square brackets, closed brackets, or box brackets:  [ ]
\newcommand{\abs}[1]{\left|#1 \right|}

\newcommand{\lerr}[2]{\left|\frac{\partial #1}{\partial #2}\right| \delta #2} % linear error
\newcommand{\qerr}[2]{\rblr{\frac{\partial #1}{\partial #2}}^2(\delta #2)^2}   % quadratic error

% Comando para usar la notación de Dirac.

\newcommand{\bra}[1]{\langle #1|}
\newcommand{\ket}[1]{|#1\rangle}
\newcommand{\braket}[2]{\langle #1|#2\rangle}
\newcommand{\mean}[1]{\left\langle #1 \right\rangle}

% Ecuaciones

\newcommand{\half}{\frac{1}{2}}
\newcommand{\eq}[1]{\begin{equation} #1 \end{equation}} % numbered equation
\newcommand{\eql}[2]{\begin{equation} #2 \label{eq:#1} \end{equation}} % numbered labeled equation
\newcommand{\eqn}[1]{\begin{equation*} #1 \end{equation*}} % no numbered equation

% Ecuaciones en array
\newcommand{\eqsnarray}[1]{\begin{eqnarray} #1 \end{eqnarray}} % equations with number in array 
\newcommand{\eqsnnarray}[2]{\begin{eqnarray*} #1 \end{eqnarray*}} % equation with no numbred in array

% TABLAS
\newcommand{\erow}{ \\ \hline}

% Fast Table ''ftable``
\newcommand{\ftable}[3]{
\begin{table}[h!]
 \centering
 \begin{tabular}{#1} \hline
 #3 \\ \hline
\end{tabular}
\caption{#2.}
\end{table}
}

% Fast array
\newcommand{\farray}[3]{
\begin{table}[h!]
 \centering
 \begin{math}
 \begin{array}{#1} \hline
 #3 \\ \hline
\end{array}
\end{math}
\caption{#2.}
\end{table}
}

% Two Fast array with common caption
\newcommand{\twofarray}[6]{
\begin{table}[h!] 
\centering

\begin{tabular}{ccccc} 
(a)  & 
\begin{math}
\begin{array}{#1}  
\hline #3 \\ \hline
\end{array}
\end{math} 
  & \hspace{0.1cm} & (b) &
\begin{math}
\begin{array}{#4} \hline
 #6 \\ \hline
\end{array}
\end{math} \\
\end{tabular}
\caption{(a) #2. (b) #5.}
\end{table}
}

% Two Fast ARRAY With Common Caption and Label

\newcommand{\twofarrayl}[8]{
\begin{table}[h!] 
\centering

\begin{tabular}{ccccc} 
(a)  & 
\begin{math}
\begin{array}{#1}  
\hline #4 \\ \hline
\end{array}
\label{#3}
\end{math} 
  & \hspace{0.1cm} & (b) &
\begin{math}
\begin{array}{#5} \hline
 #8 \\ \hline
\end{array}
\label{#7}
\end{math} \\
\end{tabular}
\caption{(a) #2. (b) #6.}
\end{table}
}

% Two Fast array with no common caption
\newcommand{\twofarrayncc}[6]{
\begin{table}[h!] 
\centering
\begin{tabular}{ccccc} 
#2  & 
\begin{math}
\begin{array}{#1}  
\hline #3 \\ \hline
\end{array}
\end{math} & \hspace{0.1cm} & 
#5 &
\begin{math}
\begin{array}{#4} \hline
 #6 \\ \hline
\end{array}
\end{math} \\
\end{tabular}
\end{table}
}

% Two Fast ARRAY with No Common Caption and Label

\newcommand{\twofarraynccl}[8]{
\begin{table}[h!] 
\centering

\begin{tabular}{ccccc} 
(a)  & 
\begin{math}
\begin{array}{#1}  
\hline #4 \\ \hline
\end{array}
\label{#3}
\end{math} 
  & \hspace{0.1cm} & (b) &
\begin{math}
\begin{array}{#5} \hline
 #8 \\ \hline
\end{array}
\label{#7}
\end{math} \\
\end{tabular}
\caption{(a) #2. (b) #6.}
\end{table}
}

\newcommand{\threefarray}[9]{
\begin{table}[h!] 
\centering

\begin{tabular}{cccccccc} 
(a)  & 
\begin{math}
\begin{array}{#1}  
\hline #3 \\ \hline
\end{array}
\end{math} 
& \hspace{0.1cm} 
& (b) &
\begin{math}
\begin{array}{#4} \hline
 #6 \\ \hline
\end{array}
\end{math}
& \hspace{0.1cm} 
& (c) &
\begin{math}
\begin{array}{#7} \hline
 #9 \\ \hline
\end{array}
\end{math} \\ 
\end{tabular}
\caption{(a) #2. (b) #5. (c) #8}
\end{table}
}

% Fast ARRAY with Side Caption
\newcommand{\farraysc}[3]{
\begin{table}[h!] 
\centering
\begin{tabular}{cc} 
#2  & 
\begin{math}
\begin{array}{#1}  
\hline #3 \\ \hline
\end{array}
\end{math}  
\end{tabular} 
\end{table}
}

%%%%%%%%%%%%%%%%%%%%%%%%%%%%%%%%%%%%%%%%%%%%%%%%%%%%%%% IMÁGENES

% A simple image
\newcommand{\img}[2]{ \includegraphics[#1]{#2} }

% Centered Numbered Image with caption
\newcommand{\cnimg}[3]{
\begin{figure}[h!]
\centering
\includegraphics[#1]{#2}
\caption{#3.}
\end{figure}
}

% Centered Numbered Image with caption and Label
\newcommand{\cnimgl}[4]{
\begin{figure}[h!]
\centering
\includegraphics[#1]{#2}
\caption{#3.}
\label{#4}
\end{figure}
}

% Two cententered numbered images
\newcommand{\twocnimg}[6]{
\begin{figure}[h!]
\centering
\subfigure[#3]{\includegraphics[#1]{#2}}
\subfigure[#6]{ \includegraphics[#4]{#5}}
\end{figure}
}

% Centered Numbered Image without caption
\newcommand{\cimg}[2]{
\begin{figure}[h!]
\centering
\includegraphics[#1]{#2}
\end{figure}
}

% ALGUNOS AMBIENTES SIMPLIFICADOS

\newcommand{\fitem}[1]{ 
\begin{itemize} 
#1
\end{itemize}
}

\newcommand{\citem}[1]{ 
\begin{compactitem} 
\item[] #1 \item[]
\end{compactitem}
}

\newcommand{\fenumerate}[1]{
\begin{enumerate}
 #1
\end{enumerate}
}

