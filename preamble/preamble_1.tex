\documentclass[10pt,a4paper,spanish]{report}

\author{Javier Carrillo Milla}

%% --- The Standard Packages ---
\usepackage{latexsym}
\usepackage{makeidx}

%% FUENTES
%\usepackage[latin1]{inputenc} %% Este paquete se selecciona para WINDOWS
\usepackage[utf8]{inputenc} %% Este paquete se selecciona para LINUX
\usepackage[spanish]{babel} % Tildes y la ñ española
%\sloppy % Mejor división de palabras

%% COLORES
\usepackage{color}
% \color{``texto o formula''} -> Cambia el color del texto a partir de este punto
% \texcolor{``color''}{texto o f�rmula}
% \pagecolor{``black``} -> Cambia el color de la p�gina a partir de este momento
% \definecolor{rosa}{rgb}{1,0.5,0.5} -> Definir un color personalizado en rgb
\usepackage{xcolor}

%% FUENTES MATEMÁTICAS
\usepackage{amsfonts}
\usepackage{amsmath}
\usepackage{amssymb}
\usepackage{amsthm}
\usepackage{upgreek}

\usepackage{mathpazo}

 
%% --- Tablas largas ---
% Con este paquete se pueden poner tablas que sean muy grandes, utilizando /begin{longtable} en
% lugar de /begin{table} ---
\usepackage{longtable}
\usepackage{multicol}
%\usepackage{multirow}
\usepackage{caption} % Extiende las posibilidades del caption

%% --- Landscape ---
\usepackage{lscape}

% Gráficos
% Se ponen los paquetes necesarios para introducir im�genes.
\usepackage{psboxit}
\usepackage{floatflt,graphicx}
\usepackage{epsfig}
\usepackage{subfigure}
\usepackage{pst-all}
\usepackage{psfrag}
\usepackage{parallel} % Escribe en paralelo

\usepackage[left=0.6in,right=0.6in,top=0.8in,bottom=0.5in,includeheadfoot]{geometry}

%%%%%%%%%%%%%%%%%%%%%%%% Filigranas varias

\newcommand{\rb}[2]{\rotatebox{#1}{#2}}
\newcommand{\there}[1]{\frac{#1}{\downarrow}} 
\newcommand{\rarrow}[2]{\rotatebox{#1}{$\xleftarrow{\text{#2}}$}} 
% Para rotar el texto. Comando ''\rb{angulo}{texto}``

\usepackage{minitoc} % Índice de capítulos 
\setcounter{minitocdepth}{2}
\setlength{\mtcindent}{0pt}
\renewcommand{\mtcfont}{\small\bf}
\renewcommand{\mtcSfont}{\small\bf}

\renewcommand{\mtctitle}{}

\usepackage{paralist} % para usar compactitem: itemize con menos espacio entre items

%%%%%%%%%%%%%%%%%%%%%%%% CABECERAS Y PIES DE PÁGINA
\usepackage{fancyhdr}
\pagestyle{fancy}

%\usepackage[Lenny]{fncychap} % Estilo de capítulos

% Para el Bjornstrup
%
%\ChNumVar{\fontsize{76}{80}\usefont{OT1}{pzc}{m}{n}\selectfont}
%\ChTitleVar{\raggedleft\Large\sffamily\bfseries}

% Para el Lenny
%\ChNameVar{\fontsize{14}{16}\usefont{OT1}{phv}{m}{n}\selectfont}
%\ChNumVar{\fontsize{60}{62}\usefont{OT1}{ptm}{m}{n}\selectfont}
%\ChTitleVar{\Huge\bfseries\rm} 
%\ChRuleWidth{1pt}

\fancyhf{}
\usepackage{calc}  % No se que es aún.

% Modificación de algunos comandos.
\renewcommand{\chaptermark}[1]{\markboth{\textbf{\thechapter. #1}}{}} % Formato para el capítulo: N. Nombre
\renewcommand{\sectionmark}[1]{\markright{\textbf{\thesection. #1}}} % Formato para la sección: N.M. Nombre
\renewcommand{\headrulewidth}{0.6pt} % Ancho de la línea horizontal bajo el encabezado
%\renewcommand{\footrulewidth}{0.6pt} % Ancho de la línea horizontal sobre el pie (que en este ejemplo está vacío)
%\setlength{\headheight}{1.5\headheight} % Aumenta la altura del encabezado en una vez y media
\setlength{\headheight}{16.0pt}

% Cabeceras
%\fancyhead[LO]{\leftmark} % En las páginas impares, parte izquierda del encabezado, aparecerá el nombre de capítulo
\fancyhead[R]{\rightmark} % En las páginas pares, parte derecha del encabezado, aparecerá el nombre de sección
\fancyhead[RO,LE]{Página \thepage} % Números de página en las esquinas de los encabezados
\fancyhead[RE,LO]{Javier Carrillo Milla} % Números de página en las esquinas de los encabezados

% Pie de página
%\fancyfoot[L]{}
%\fancyfoot[LO]{\today}
%\fancyfoot[C]{\thepage} %AQUI SE PONE POR DEFECTO LA PAGINA, EN EL PIE DE


%%%%%%%%%%%%%%%%%%%%%%%%%%%%%%%%%%%%%%%%%%%%%%%%%%%%%%% FOOTNOTES

\usepackage{footnote}

% Tamaño del footnote
\renewcommand{\footnoterule}{\vspace*{-3pt}
  \noindent\rule{15cm}{1pt}\vspace*{2.6pt}}

% Estilos para el footnote
\renewcommand{\thefootnote}{(\arabic{footnote})} 	 % 	Numeración arábiga: 1, 2, 3...
%\renewcommand{\thefootnote}{\roman{footnote}} 	 %	Numeración romana en minúsculas: i, ii, iii...
%\renewcommand{\thefootnote}{\Roman{footnote}} 	 %	Numeración romana en ayúsculas: I, II, III...
%\renewcommand{\thefootnote}{\alph{footnote}} 	 %      Numeración alfabética en minúsculas a, b, c...
%\renewcommand{\thefootnote}{\Alph{footnote}} 	 %	Numeración alfabética en mayúsculas: A, B, C...
%\renewcommand{\thefootnote}{\fnsymbol{footnote}} %	No números, sino símbolos diversos

%%%%%%%%%%%%%%%%%%%%%%%%%%%%%%%%%%%%%%%%%%%%%%%%%%%%%%% Bibliografía
\usepackage[square,authoryear,sort&compress]{natbib}

%% Custom bibliography style defined using the Makebst utility
%% that comes with the natbib package.

\bibliographystyle{bib/bibleo}

%%%%%%%%%%%%%%%%%%%%%%%%%%%%%%%%%%%%%%%%%%%%%%%%%%%%%%% Comandos personalizados
% Estos paquetes se usan para poner nombre a diversos objetos.
\addto\captionsspanish{%
%  \def\prefacename{Prefacio}%
%  \def\refname{Referencias}%
  \def\abstractname{Resumen}
  \def\bibname{Bibliograf\'{\i}a}%
  \def\chaptername{Apartado}%
%  \def\appendixname{Ap\'endice}%
%  \def\listfigurename{\'Indice de figuras}%
  %\def\listtablename{\'Indice de cuadros}%
  \def\listtablename{\'Indice de Tablas}%
%  \def\indexname{\'Indice alfab\'etico}%
%  \def\figurename{Figura}%
  %\def\tablename{Cuadro}%
  \def\tablename{Tabla}%
%  \def\partname{Parte}%
%  \def\enclname{Adjunto}%
%  \def\ccname{Copia a}%
%  \def\headtoname{A}%
%  \def\pagename{P\'agina}%
%  \def\seename{v\'ease}%}
%  \def\alsoname{v\'ease tambi\'en}%
   \def\theoremname{Teorema}
   \def\definitionname{Definici\'on}
   \def\lemmaname{Lema}
   \def\corolaryname{Corolario}
%  \def\proofname{Demostraci\'on}%
%  \def\glossaryname{Glosario}
}

% Boxedminipage
\usepackage{boxedminipage}
\setlength{\fboxrule}{1.4pt}
\setlength{\fboxsep}{1.5pt}