%%%%%%%%%%%%%%%%%%%%%%%%%%%%%%%%%%%%%%%%%%%%%%%%%%%%%%%%%%%%%%%%%%%%%%%%%%%%%%%%%%%
%% Notes:
%% ----------------------------
%% Caution! the order matters to compile! Right know one learn that
%% -> First graphics packages. 
%% -> Afterwards \documentclass... 
%% maintain the order if don't want a lot of headaches! :)
%%%%%%%%%%%%%%%%%%%%%%%%%%%%%%%%%%%%%%%%%%%%%%%%%%%%%%%%%%%%%%%%%%%%%%%%%%%%%%%%%%%

% -----------------------------------------------
% The Standard Packages
% -----------------------------------------------
\usepackage[utf8x]{inputenc}
\usepackage[T1]{fontenc}
\usepackage[spanish]{babel}  % Language
%\usepackage{latexsym}
% -----------------------------------------------
% Related to Graphics
% -----------------------------------------------
%\usepackage{graphicx}
%\usepackage{wrapfig} % To place figures around the text
\usepackage[pdftex]{graphicx}
\usepackage{subfigure}
%\usepackage[sub,ovp]{psfragx}
%\usepackage{overpic,color}

% Captions
\usepackage{caption}

% -----------------------------------------------
% Related to -> Text
% -----------------------------------------------
%% Whole Text Fonts
%\usepackage{kpfonts}
%\usepackage{wasysym}  % Strange symbols
\usepackage{mathpazo}  % Font used in the whole text
\usepackage{ulem}      % Different types of underline as wave underline
\usepackage{cancel}    % To strike down/up things on text/math
%\cancel{text to cancel} draws a diagonal line (slash) through its argument
%\bcancel{text to cancel} uses the negative slope (a backslash)
%\xcancel{text to cancel} draws an X (actually \cancel plus \bcancel)
%\cancelto{〈value〉}{〈expression〉} draws a diagonal arrow through the 〈expression〉, pointing to the 〈value〉
% (math-mode only)
\usepackage{eurosym}  % euro symbol!
% Mathematical fonts and environments
\usepackage{amsfonts}
\usepackage{amsmath}
\usepackage{amssymb}
\usepackage{amsthm}
\usepackage{dsfont}
% Especific fonts
\def\dbar{{\mathchar'26\mkern-10mu d}}  % Inexact differential


% -----------------------------------------------
%% Related to -> Page Formating
% -----------------------------------------------
% Page margins
\usepackage[left=0.1\textwidth,right=0.1\textwidth,top=0.05\textwidth,bottom=0.13\textwidth,includehead]{geometry} 
\setlength{\parindent}{0pt}  % Paragraph indentation for the hole document
%\sloppy                      % Better division between words

% Headers and footers
\usepackage{fancyhdr}
\pagestyle{fancy}
\fancyfoot{}
%\usepackage[Conny]{fncychap} % Predefined chapter style
%\fancyhf{}
%\usepackage{calc}  % Don't remember what is this

% HEADERS
%\fancyhead[C]{\sectionmark} % En las páginas impares, parte izquierda del encabezado, 
%\fancyhead[RO,LE]{- \thepage -} % Números de página en las esquinas de los encabezados
%\fancyhead[RO,LE]{Javier Carrillo Milla} % Números de página en las esquinas de los encabezados

% FOOTERS
\fancyfoot{}
%\fancyfoot[L]{}
%\fancyfoot[LO]{\today}
\fancyfoot[RO,LE]{$\boxed{\text{\thepage}}$}
%\fancyfoot[LE]{$\boxed{\text{\thepage}}$}

% FOOTNOTES
\usepackage{footnote}
\renewcommand{\footnoterule}{\vspace*{-3pt} % Footnote format
  \noindent\rule{15cm}{1pt}\vspace*{2.6pt}}
\renewcommand{\thefootnote}{[\arabic{footnote}]} % Styles for the footnote
%\renewcommand{\thefootnote}{\roman{footnote}}
%\renewcommand{\thefootnote}{\Roman{footnote}} 	 %	Numeración romana en ayúsculas: I, II,
%\renewcommand{\thefootnote}{\alph{footnote}} 	 %      Numeración alfabética en minúsculas a, b,
%\renewcommand{\thefootnote}{\Alph{footnote}} 	 %	Numeración alfabética en mayúsculas: A, B,
%\renewcommand{\thefootnote}{\fnsymbol{footnote}} %	No números, sino símbolos diversos

% Columns and environments
\usepackage{longtable}       % To include long tables (tabulars)
%\renewcommand{\tabcolsep}{1cm}
\renewcommand{\arraystretch}{1}

\usepackage{parcolumns}      % enables two-column style in the middle of a document
\usepackage{multicol}        % Multiple columns on a tabular
\usepackage{multirow}
\setlength{\columnsep}{1.5cm}
\setlength{\columnseprule}{2pt}
% Float figures in multicols
\newenvironment{Figure}
  {\par\medskip\noindent\minipage{\linewidth}}
  {\endminipage\par\medskip}

\usepackage{paralist}        % Contains compactitem: like itemize but less space between items.
%\usepackage{parallel}
% Minipage & Boxedminipage
\usepackage{boxedminipage}
\setlength{\fboxrule}{1.2pt}
\setlength{\fboxsep}{3.5pt}
% List

% Coloring, Sizing, Orientation, Deforming the text
% Fontcolors
\usepackage[dvinames]{xcolor}
%-> Usage
% --
% \color{text or eq} -> Changes text from this point on.
% \texcolor{color}{text or eq} - > Changes text under the curly brackets.
% \pagecolor{black} -> Changes the background color of your page.
% \definecolor{pink}{rgb}{1,0.5,0.5} -> Custom color from rgb.
%\usepackage{xcolor}
\usepackage{rotating}        % To rotate text
\usepackage{lscape}          % landscape page format

\usepackage{listings}        % Code Highlighting
\lstset{                     % Options for listings
language=R, 
basicstyle=\scriptsize\ttfamily\color{blue}, 
commentstyle=\ttfamily\color{gray}, 
numbers=left, 
numberstyle=\ttfamily\color{red}\scriptsize, 
stepnumber=1, 
numbersep=5pt, 
backgroundcolor=\color{white}, 
showspaces=false, 
showstringspaces=false, 
showtabs=false, 
xleftmargin=0.1\textwidth,
xrightmargin=0.1\textwidth,
frame=single,    
   %framexleftmargin=17pt,
   %framexrightmargin=5pt,
   %framexbottommargin=4pt,
tabsize=4, 
captionpos=b, 
breaklines=true, 
breakatwhitespace=false, 
title=\lstname, 
escapeinside={}, 
keywordstyle={}, 
morekeywords={} 
} 

% -----------------------------------------------
% Indexing & Bibliography
% -----------------------------------------------
\usepackage{makeidx}
%\usepackage[square,authoryear,sort&compress]{natbib}
%% Custom bibliography style defined using the Makebst utility
%% that comes with the natbib package.
%\bibliographystyle{bib/bibleo}

% -----------------------------------------------
% Related to -> User Defined Environments
% -----------------------------------------------
%% Definition of the environments.
% Format: \newtheorem{<your name>}{<Definition, Theorem or Lemma>}
%\newtheorem{theorem}{Theorem}[section]
%\newtheorem{definition}{Definition}[section]
%\newtheorem{lemma}{Lemma}[section]