\documentclass[a4paper,10pt]{article}
%%%%%%%%%%%%%%%%%%%%%%%%%%%%%%%%%%%%%%%%%%%%%%%%%%%%%%%%%%%%%%%%%%%%%%%%%%%%%%%%%%%
%% Notes:
%% ----------------------------
%% Caution! the order matters to compile! Right know one learn that
%% -> First graphics packages. 
%% -> Afterwards \documentclass... 
%% maintain the order if don't want a lot of headaches! :)
%%%%%%%%%%%%%%%%%%%%%%%%%%%%%%%%%%%%%%%%%%%%%%%%%%%%%%%%%%%%%%%%%%%%%%%%%%%%%%%%%%%

% -----------------------------------------------
% The Standard Packages
% -----------------------------------------------
\usepackage[utf8x]{inputenc}
\usepackage[T1]{fontenc}
\usepackage[spanish]{babel}  % Language
%\usepackage{latexsym}
% -----------------------------------------------
% Related to Graphics
% -----------------------------------------------
%\usepackage{graphicx}
%\usepackage{wrapfig} % To place figures around the text
\usepackage[pdftex]{graphicx}
\usepackage{subfigure}
%\usepackage[sub,ovp]{psfragx}
%\usepackage{overpic,color}

% Captions
\usepackage{caption}

% -----------------------------------------------
% Related to -> Text
% -----------------------------------------------
%% Whole Text Fonts
%\usepackage{kpfonts}
%\usepackage{wasysym}  % Strange symbols
\usepackage{mathpazo}  % Font used in the whole text
\usepackage{ulem}      % Different types of underline as wave underline
\usepackage{cancel}    % To strike down/up things on text/math
%\cancel{text to cancel} draws a diagonal line (slash) through its argument
%\bcancel{text to cancel} uses the negative slope (a backslash)
%\xcancel{text to cancel} draws an X (actually \cancel plus \bcancel)
%\cancelto{〈value〉}{〈expression〉} draws a diagonal arrow through the 〈expression〉, pointing to the 〈value〉
% (math-mode only)
\usepackage{eurosym}  % euro symbol!
% Mathematical fonts and environments
\usepackage{amsfonts}
\usepackage{amsmath}
\usepackage{amssymb}
\usepackage{amsthm}
\usepackage{dsfont}
% Especific fonts
\def\dbar{{\mathchar'26\mkern-10mu d}}  % Inexact differential


% -----------------------------------------------
%% Related to -> Page Formating
% -----------------------------------------------
% Page margins
\usepackage[left=0.1\textwidth,right=0.1\textwidth,top=0.05\textwidth,bottom=0.13\textwidth,includehead]{geometry} 
\setlength{\parindent}{0pt}  % Paragraph indentation for the hole document
%\sloppy                      % Better division between words

% Headers and footers
\usepackage{fancyhdr}
\pagestyle{fancy}
\fancyfoot{}
%\usepackage[Conny]{fncychap} % Predefined chapter style
%\fancyhf{}
%\usepackage{calc}  % Don't remember what is this

% HEADERS
%\fancyhead[C]{\sectionmark} % En las páginas impares, parte izquierda del encabezado, 
%\fancyhead[RO,LE]{- \thepage -} % Números de página en las esquinas de los encabezados
%\fancyhead[RO,LE]{Javier Carrillo Milla} % Números de página en las esquinas de los encabezados

% FOOTERS
\fancyfoot{}
%\fancyfoot[L]{}
%\fancyfoot[LO]{\today}
\fancyfoot[RO,LE]{$\boxed{\text{\thepage}}$}
%\fancyfoot[LE]{$\boxed{\text{\thepage}}$}

% FOOTNOTES
\usepackage{footnote}
\renewcommand{\footnoterule}{\vspace*{-3pt} % Footnote format
  \noindent\rule{15cm}{1pt}\vspace*{2.6pt}}
\renewcommand{\thefootnote}{[\arabic{footnote}]} % Styles for the footnote
%\renewcommand{\thefootnote}{\roman{footnote}}
%\renewcommand{\thefootnote}{\Roman{footnote}} 	 %	Numeración romana en ayúsculas: I, II,
%\renewcommand{\thefootnote}{\alph{footnote}} 	 %      Numeración alfabética en minúsculas a, b,
%\renewcommand{\thefootnote}{\Alph{footnote}} 	 %	Numeración alfabética en mayúsculas: A, B,
%\renewcommand{\thefootnote}{\fnsymbol{footnote}} %	No números, sino símbolos diversos

% Columns and environments
\usepackage{longtable}       % To include long tables (tabulars)
%\renewcommand{\tabcolsep}{1cm}
\renewcommand{\arraystretch}{1}

\usepackage{parcolumns}      % enables two-column style in the middle of a document
\usepackage{multicol}        % Multiple columns on a tabular
\usepackage{multirow}
\setlength{\columnsep}{1.5cm}
\setlength{\columnseprule}{2pt}
% Float figures in multicols
\newenvironment{Figure}
  {\par\medskip\noindent\minipage{\linewidth}}
  {\endminipage\par\medskip}

\usepackage{paralist}        % Contains compactitem: like itemize but less space between items.
%\usepackage{parallel}
% Minipage & Boxedminipage
\usepackage{boxedminipage}
\setlength{\fboxrule}{1.2pt}
\setlength{\fboxsep}{3.5pt}
% List

% Coloring, Sizing, Orientation, Deforming the text
% Fontcolors
\usepackage[dvinames]{xcolor}
%-> Usage
% --
% \color{text or eq} -> Changes text from this point on.
% \texcolor{color}{text or eq} - > Changes text under the curly brackets.
% \pagecolor{black} -> Changes the background color of your page.
% \definecolor{pink}{rgb}{1,0.5,0.5} -> Custom color from rgb.
%\usepackage{xcolor}
\usepackage{rotating}        % To rotate text
\usepackage{lscape}          % landscape page format

\usepackage{listings}        % Code Highlighting
\lstset{                     % Options for listings
language=R, 
basicstyle=\scriptsize\ttfamily\color{blue}, 
commentstyle=\ttfamily\color{gray}, 
numbers=left, 
numberstyle=\ttfamily\color{red}\scriptsize, 
stepnumber=1, 
numbersep=5pt, 
backgroundcolor=\color{white}, 
showspaces=false, 
showstringspaces=false, 
showtabs=false, 
xleftmargin=0.1\textwidth,
xrightmargin=0.1\textwidth,
frame=single,    
   %framexleftmargin=17pt,
   %framexrightmargin=5pt,
   %framexbottommargin=4pt,
tabsize=4, 
captionpos=b, 
breaklines=true, 
breakatwhitespace=false, 
title=\lstname, 
escapeinside={}, 
keywordstyle={}, 
morekeywords={} 
} 

% -----------------------------------------------
% Indexing & Bibliography
% -----------------------------------------------
\usepackage{makeidx}
%\usepackage[square,authoryear,sort&compress]{natbib}
%% Custom bibliography style defined using the Makebst utility
%% that comes with the natbib package.
%\bibliographystyle{bib/bibleo}

% -----------------------------------------------
% Related to -> User Defined Environments
% -----------------------------------------------
%% Definition of the environments.
% Format: \newtheorem{<your name>}{<Definition, Theorem or Lemma>}
%\newtheorem{theorem}{Theorem}[section]
%\newtheorem{definition}{Definition}[section]
%\newtheorem{lemma}{Lemma}[section]
\input{/home/javier/commands_nld}
%%%%%%%%%%%%%%%%%%%%%%%%%%%%%%%%%%%%%%%%%%%%%%%%%%%%%%% 
% Here one can change the predefined names
%%%%%%%%%%%%%%%%%%%%%%%%%%%%%%%%%%%%%%%%%%%%%%%%%%%%%%% 

%\addto\captions{%
%  \def\prefacename{Prefacio}%
%  \def\contentsname{\'Indice de contenidos}%
  \def\abstractname{Resumen}
%  \def\bibname{Bibliograf\'{\i}a}%
   \def\chaptername{Cap\'itulo}%
%  \def\appendixname{Ap\'endice}%
%  \def\listfigurename{\'Indice de figuras}%
%  \def\listtablename{\'Indice de cuadros}%
%  \def\listtablename{\'Indice de Tablas}%
%  \def\indexname{\'Indice alfab\'etico}%
  \def\figurename{Figura}%
%  \def\tablename{Cuadro}%
%  \def\tablename{Tabla}%
%  \def\partname{Parte}%
%  \def\enclname{Adjunto}%
%  \def\ccname{Copia a}%
%  \def\headtoname{A}%
%  \def\pagename{P\'agina}%
%  \def\seename{v\'ease}%}
%  \def\alsoname{v\'ease tambi\'en}%
%  \def\theoremname{Teorema}
%  \def\definitionname{Definici\'on}
%  \def\lemmaname{Lemma}
%  \def\corolaryname{Corolario}
%  \def\proofname{Demostraci\'on}%
%  \def\glossaryname{Glosario}
%}

\title{Manejo de instrumentos}
\author{Javier Carrillo Milla \\ Emilio Jos\'e Garc\'ia Lorente}
\date{\today}

\begin{document}
\maketitle

\begin{abstract}
Esta pr\'actica est\'a orientada a familiarizarse con los instrumentos que se utilizar\'an en
pr\'acticas posteriores. Para ello se pide el an\'alisis del voltaje de salida y entrada en el
osciloscopio, una vez montado el circuito en cuesti\'on. 
\end{abstract}

\section*{Material de Prácticas}

En el laboratorio se dispuso de los siguientes componentes para el circuito (además de las fuentes, polímetro, etc.)

\begin{itemize}
 \item $R_1 = 11.91$ $k\Omega$
 \item $R_2 = 15.93$ $k\Omega$ 
 \item $C =2 10$ $nF$
\end{itemize}

Además se sabe que el condensador de la sonda tiene un valor $C_{son} \approx pF$ 

\section{Montaje 1}

\subsection{Trabajo teórico}

Expresar en forma de función matemática la señal de entrada $V_i(t)$ que se presenta a continuación 

\begin{figure}[h!]
\centering
\subfigure[$v_i(t)$]{
 \includegraphics[scale=.8]{img/vi_1}
}
\subfigure[Circuito a estudiar]{
\includegraphics[scale=.8]{img/montaje_1}
}
\end{figure}

junto con la de salida del circuito.\\
\hl

Tal y como está indicada la tierra \verb'GND' y las divisiones en la figura la
$v_i(t)$ corresponde a 

\begin{equation}
v_i(t) = sen(\omega t) + 1, \quad \omega = \frac{2\pi}{T}  = \frac{2\pi}{0.5} = 4\cdot 10^3\pi rad/s
\end{equation}

Para obtener $v_o(t)$ habría que resolver la ecuación del circuito. Que corresponde a 

\begin{equation}
 v_i(t) = i(t) R_1 + v_c(t) + i(t) R_2 
\end{equation}

donde $v_o(t) = i(t) R_2 = C R_2 \frac{dv_c(t)}{dt}$, luego habrá que calcular $v_c(t)$. 

\begin{equation}\left.
\begin{array}{lll}
sen(\omega t) + 1 &=& i(t)(R_1 + R_2) + v_c(t) \\
C(R_1+R_2)\frac{dv_c(t)}{dt} + v_c(t) &=& sen(\omega t) + 1 \\
\end{array} \right\}, \quad 
v_c(t) = 1 - e^{-\frac{t}{\tau }} + 
\frac{   
e^{-\frac{t}{\tau }}   
\left[\tau  \omega -e^{t/\tau } \tau  \omega  cos(t\omega)+e^{t/\tau } sin(t \omega )\right]
}{ 1+\tau ^2 \omega^2 }
\end{equation}

$\tau = C(R_1 + R_2)$ y $A$ es la constante a determinar con la condidición inicial de que el condensador esté
descargado en $t = 0$, es decir $v_c(0) = 1$ (Voltaje para $v_i(0) = 1$) $\Rightarrow A = \frac{\tau  \omega}{1+\tau ^2
\omega ^2} - 1$. De esta forma, resulta que $v_o(t)$ es

\begin{equation}
 v_o(t) = C R_2 \frac{dv_c(t)}{dt}= C R_2\frac{ e^{-\frac{t}{\tau }} \left(1+\tau  \omega  (-1+\tau 
\omega )+e^{t/\tau } \tau  \omega  (\text{Cos}[t \omega ]+\tau  \omega  \text{Sin}[t \omega ])\right)}{\tau +\tau ^3
\omega ^2}
\end{equation}

que representada junto a $v_i(t)$ queda

\begin{figure}[h!]
 \centering 
 \includegraphics[scale=.5]{img/montaje_1_img}
\end{figure}

Las amplitudes, el desfase son (se busca el máximo de cada función el instante $t$ en que ocurre, luego se restan los
tiempos en los que la amplitud era máxima y se obtiene el desfase) y los valores medios son

\begin{equation}
 \begin{array}{l}
  |v_i(t)| = 2 V, \quad |v_i(t)| = 0.3703 V \\
  \delta = 72.6 \mu s \\
  \mean{v_i(t)} = \frac{1}{T} \int^T_0 v_i(t) dt = 1 V \\
  \mean{v_o(t)} = \frac{1}{T} \int^T_0 v_o(t) dt = 0.0337846 V
 \end{array}
\end{equation}

\subsection{Trabajo del laboratorio}

En el laboratorio se reprodujeron las señal $v_i(t)$ anterior, y se observó la correspondiente para la resistencia. Una
vez hecho esto se pedía observarla señal en modo AC y DC, en modo AC era justo lo que se ve arriba. Sin embargo en modo
DC se eliminaba el offset introducido por la fuetne para obtener $v_i(t)$ y el desfase entre ambas señales. \\ 

Los valores obtenidos en el laboratorio para $v_o(t)$ difieren poco de lo aquí calculado. 

\section{Montaje 2}

\subsection{Trabajo teórico y del laboratorio}

Analícese siguiente circuito

\begin{figure}[h!]
 \centering
 \includegraphics[scale=.5]{img/montaje_2}
\end{figure}

donde $v_i(t)$ si se aplica un escalón de tensión de entrada entre $0$ y $2$ $V$. 

\begin{enumerate}[(i)]
 \item Comportamiento transitorio.
 \item Expresión de $V_o(t)$ en este caso y representarla
 \item Encontrar el tiempo característico. 
 \item Obtener el tiempo de subida (tiempo entre que la señal pasa por el 10\% y el 90\% de su valor máximo. Relacionar
esto con el tiempo característico. Determinar si la capacidad de la sonda influye o no en estas medidas.  
\end{enumerate}

||||||||||||||||||||\\

Como había que ver tanto la subida como la bajada en el laboratorio se utilizo una señal cuadrada de periodo $T =
2\pi/\omega$. Esto nos lleva a resolver la ecuaciones diferenciales siguientes

\begin{equation}
\begin{array}{l}
 \left.
 \begin{array}{l} \displaystyle
 C(R_1+R_2)\frac{dv_{o1}(t)}{dt} + v_{o1}(t) = 2 \\ \displaystyle
 v_{o1}(0) = 0
 \end{array} \right\} \Rightarrow v_{o1}(t) = 2 e^{-\frac{t}{\tau }} \left(-1+e^{t/\tau }\right); \\ \\
 \left.
 \begin{array}{l} \displaystyle
 C(R_1+R_2)\frac{dv_{o2}(t)}{dt} + v_{o2}(t) = 0 \\ \displaystyle
 v_{o2}(T) = 2
 \end{array} \right\} \Rightarrow v_{o2}(t) = 2 e^{-\frac{t}{\tau }+\frac{T}{\tau }} 
\end{array}
\end{equation}

en principio $ C = \tilde C + C_{osc}$ pero al ser la sonda del osciloscopio tan pequeña no va a contribuir a los
resultados, por lo que podemos dejar $\tilde C = C$. \\

El transitorio corresponde por tanto a $v_{o1}(t)$ siendo este

\begin{figure}[h!]
 \centering
 \includegraphics[scale=.4]{img/montaje_2_trans}
 \label{fig:mont_2_trans}
\end{figure}

A la salida encontraremos la siguiente señal (que seprenta junto al pulso cuadrado $v_i(t)$)

\begin{figure}[h!]
 \centering  
  \includegraphics[scale=.6]{img/montaje_2_img} 
\end{figure}

Como vemos, la señal de salida tiene un tiempo de subida y bajada que depende de la capacidad del condesador en mayor
medida (por el orden $10^{-9}$).\\

Para determinar el tiempo de subida hemos de ver qué tiempo transcurre entre que la señal tiene el 90\% y el 10\%
de la señal. Para ello se realiza la siguiente tabla de valores

\begin{equation}
\begin{array}{c|c}
 t & v_o(t) \\ \hline
 \tau  & 1.26424 \\
 \text{3$\tau $} & 1.90043 \\
 \text{5$\tau $} & 1.98652 \\
 \text{7$\tau $} & 1.99818 \\
 \text{11$\tau $} & 1.99997 \\
 \text{13$\tau $} & 2.
\end{array}
\end{equation}

Como vemos el valor máximo de la señal corresponde a $2 V$ (tal y como hemos impuesto en las condiciones de contorno).
Así el $90\%$ de la señal serán $1.8$ $V$ y el $10\%$ $0.2$ $V$. Los tiempos correspondientes serán pues 

\begin{equation} \left.
\begin{array}{l}
\tau = 55.68 \mu s
1.8 V \to t_{90\%} =   11.7 \tau = 0.65 \mu s
0.2 V \to t_{90\%} =   2.6 \tau = 0.11 \mu s
\end{array} \right\} \to t_{subida}  \approx (0.51 \pm 0.02) \mu s
\end{equation}

En el laboratorio se obtuvo un valor de $ t_{subida} = (0.47 \pm 0.01) \mu s$. \\

Tanto las gráficas como los valores fueron similares en el laboratorio. 



\end{document}
